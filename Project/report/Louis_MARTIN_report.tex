\documentclass[a4paper,11pt]{article}

\usepackage[utf8]{inputenc} % allow utf-8 input
\usepackage{hyperref}       % hyperlinks
\usepackage{url}            % simple URL typesetting
\usepackage{booktabs}       % professional-quality tables
\usepackage{amsfonts}       % blackboard math symbols
\usepackage{nicefrac}       % compact symbols for 1/2, etc.
\usepackage{microtype}      % microtypography
\usepackage{graphicx}       % include graphics
\usepackage{subcaption}     % subfigures
\usepackage{float}          % placement of floats
\usepackage{fancyhdr}       % head notes and foot notes

\graphicspath{ {images/} }





\title{P3 - Dictionary learning, application to denoising and inpainting}


\author{
  Louis Martin\\
  \href{mailto:louis.martin@student.ecp.fr}{\tt louis.martin@student.ecp.fr}
}

\pagestyle{fancy}
\fancyhf{}
\lhead{Louis MARTIN}
\rhead{Sparsity and Compressed Sensing: Project}

\rfoot{Page \thepage}

\begin{document}

\maketitle
\pagebreak

\begin{abstract}
% Abstract (½ page): 
% What problem(s) is studied ? 
% Why is it relevant ? 
% What solution(s) is proposed ? 
% Which contributions (theory, numerics, etc) ? 

\end{abstract}

\section{Introduction}
% Introduction (~3 pages) : 
% Presentation of the problem(s). 
% Previous works (at least a few citations). If relevant, include things that you have seen during the MVA course (or possibly other courses). 
% Contributions. Why is the studied method different/better/worse/etc. than existing previous works. 

Decomposing a vector on a sparse basis is good because of: 
% Why do we want sparsity, i.e. a small number of non-zero coefficients ?
% 1) Compression
% 2) Theorem in the first lecture -> good compression coefficients => good denoising
% 3) Natural images tend to have a small number of non-zero coefficients.
% Complex images like random images (or lot of noise) or trees will have a higher number of coefficient.
% Sparsity is the a good indicator of natural images
% TODO: citation with author and date
The choice of the basis is important.
One can use a predefined basis which will be the same for all tasks and datasets as the Fourier basis or the wavelet basis \cite{mallat1999}.
% TODO: talking of 'basis' is abusive because it is often overcomplete
Another approach is to learn the basis, i.e. to learn a dictionary which gives good results in several tasks has 
%%%%%%%%%% Taken from \cite{matrixfactorization}
recently led to state-of-the-art
results in numerous low-level signal processing tasks such as image denoising (Elad and Aharon,
2006; Mairal et al., 2008b), texture synthesis (Peyre, 2009) and audio processing (Grosse et al., ´
2007; Fevotte et al., 2009; Zibulevsky and Pearlmutter, 2001), as well as higher- ´ level tasks such as
image classification (Raina et al., 2007; Mairal et al., 2008a, 2009b; Bradley and Bagnell, 2009;
Yang et al., 2009),
%%%%%%%%%%
The basis is not orthogonal -> more flexibility
The learnt basis look like wavelets / Gabor filters but are tuned to the data -> better accuracy

\section{MAIN BODY}
% Main body (~10 pages) : 
% Presentation of the method(s). 
% Theoretical guarantees. 
% Numerics. 

\section{Conclusion}
% Conclusion and perspective (~1 page)
% Summary of the result obtained: pros and cons (limitation, problems, error in the articles, etc)
% Possible improvement/extension
First filters of neural networks -> looks like a dictionary



\begin{thebibliography}{9}
% Bibliography (~1 page)
\bibitem{matrixfactorization}
  J. Mairal, F. Bach, J.Ponce, G. Sapiro,
  \emph{Online Learning for Matrix Factorization and Sparse Coding},
  Journal of Machine Learning Research 11 19-60, 2010.
  % http://www.di.ens.fr/~fbach/mairal10a.pdf

\bibitem{ksvd}
  M. Aharon, M. Elad, A. Bruckstein,
  \emph{K-SVD: An Algorithm for Designing Overcomplete Dictionaries for Sparse Representation},
  IEEE transactions on signal processing, VOL. 54, NO. 11, November 2006.
  % http://www.cs.technion.ac.il/~elad/publications/journals/2004/32_KSVD_IEEE_TSP.pdf
 
 %TODO: beautify
\bibitem{mallat1999}
. Mallat. A Wavelet Tour of Signal Processing, Second Edition. Academic Press, New York,
September 1999.
\end{thebibliography}
\end{document}

% Information for the project report: 
% If possible, write the report in english. 
% You must bring a printed version with your to give me before the presentation. This is a strict deadline. 
% Using Latex is highly recommended.
% Roughly 15 pages (A4, 11pt font, single column), including roughly one page of bibliography.
% Should be similar to a (short) scientific article in an applied mathematics journal. 
% No source code. 
% When presenting the numerics, give all parameters, so that the results are reproducible. 
% It should not be a rewriting of the original article. You should only report about what you have done, and only explain the theory that is relevant to explain the numerics you have done.
% Suggestion of structure for the report: 
% Abstract (½ page): 
% What problem(s) is studied ? 
% Why is it relevant ? 
% What solution(s) is proposed ? 
% Which contributions (theory, numerics, etc) ? 
% Introduction (~3 pages) : 
% Presentation of the problem(s). 
% Previous works (at least a few citations). If relevant, include things that you have seen during the MVA course (or possibly other courses). 
% Contributions. Why is the studied method different/better/worse/etc. than existing previous works. 
% Main body (~10 pages) : 
% Presentation of the method(s). 
% Theoretical guarantees. 
% Numerics. 
% Conclusion and perspective (~1 page)
% Summary of the result obtained: pros and cons (limitation, problems, error in the articles, etc)
% Possible improvement/extension
% Bibliography (~1 page)
% 
% 
% P3 - Dictionary learning, application to denoising and inpainting
% [Louis MARTIN]
% http://www.numerical-tours.com/matlab/sparsity_4_dictionary_learning/
% http://www.numerical-tours.com/matlab/sparsity_5_dictionary_learning_denoising/
% http://www.cs.technion.ac.il/~elad/publications/journals/2004/32_KSVD_IEEE_TSP.pdf
% http://www.di.ens.fr/~fbach/mairal10a.pdf
% 
% 
% Very useful to have a redundant dictionary i.e. more words than each word dimension.
% 
 
% En gros implementer ce qu'ils disent dans les papiers.
% Jouer avec les parametres et essayer de voir ce qui marche bien.
% Parametres:
% taille des patch
% choix du seuil
% param de moyennage (param magique)
% 
% Appliquer ça à l'inpainting:
% faire le dictionnary learning direct sur l'image avec des trous. Ca marche mieux
% que de le faire sur une image sans trous.
% 
% 
% 
